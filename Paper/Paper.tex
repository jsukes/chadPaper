\documentclass[prl,aps,epsf,twocolumn,superscriptaddress]{revtex4-1}

\usepackage{amsthm,mathtools,soul}
\usepackage{graphicx,subfigure,morefloats,multirow}
\usepackage{siunitx}
\sisetup{%
	detect-all,
	per-mode=symbol}

\begin{document}

\title{Quantitative Analysis of Laser and Histotripsy Generated Bubbles\\ in Viscoelastic Media}

\author{Chad T. Wilson}
\email{wilsonct@umich.edu}
\thanks{Corresponding author}
\affiliation{Current Affil.: Department of Biomedical Engineering, University of Michigan, Ann Arbor, Michigan 48105, USA} 

\author{Timothy L Hall}
%\email{hallt@umich.edu}
\affiliation{Current Affil.: Department of Biomedical Engineering, University of Michigan, Ann Arbor, Michigan 48105, USA} 

\author{Eric Johnsen}
\affiliation{Current Affil.: Department of Biomedical Engineering, University of Michigan, Ann Arbor, Michigan 48105, USA} 

\author{Christian Franck}
%\email{zhenx@umich.edu}
\affiliation{Current Affil.: Department of Mechanical Engineering, Brown University, Providence. Rhode Island 02912, USA} 

\author{Zhen Xu}
%\email{zhenx@umich.edu}
\affiliation{Current Affil.: Department of Biomedical Engineering, University of Michigan, Ann Arbor, Michigan 48105, USA} 

\author{Jonathan R. Sukovich}
\email{jsukes@umich.edu}
\thanks{Corresponding author}
\affiliation{Current Affil.: Department of Biomedical Engineering, University of Michigan, Ann Arbor, Michigan 48105, USA} 

\begin{abstract}
Presented here are data sets corresponding to the expansion and collapse of single laser and histotripsy generated bubbles in media with varying viscoelastic properties. Observation and data collection of the expansion and collapse lifetime of individual bubbles was accomplished using multiflash high speed photography. Upon analysis of the data acquired it was observed that the maximum size and lifetime of a single bubble decreases as viscosity of the material increases, and the growth and collapse phase times of the bubble deviates from the Rayleigh prediction curve. A goal of the following report is to begin to quantify the relationship between bubble dynamics and surrounding media, and provide the community with experimentally gained data to be used in computational models for the growth and ensuing collapse of nucleated bubbles in viscoelastic media.
\end{abstract}

\maketitle

\section{Introduction}
In this paper we present images of laser and ultrasound nucleated single bubbles taken during the entirety of each bubble's existence. The goal of the experiments was to establish techniques to allow for the acquisition of time-resolved images of bubbles' formation and disintegration in water and a variety of viscoelastic gels as well as to compare these results to previous testing, modelling, and theoretical predictions. Data received experimentally regarding the growth dynamics of single bubbles will be used to quantify the rheological properties of different viscoelastic classes of materials, in an effort to predict mediums by evaluating bubble dynamics. Two methods for cavitation being compared within the data set are laser and intrinsic threshold histotripsy. Histotripsy has been a term coined in recent decades used to describe the process of utilizing high frequency ultrasound waves to create regions of low pressure where molecules such as water mechanically separate with such force as to create a near-vacuum bubble \cite{maxwell2013probability}. Behaviors related to this expansion and subsequent collapse of such bubbles are the focus of this study. Histotripsy has been substantially investigated for use as a non-invasive tissue ablation therapy technique; however, the mechanisms of damage are poorly understood and models of viscoelastic media are not well developed. In an effort to counteract this trend and attempt to quantify the tissue destruction mechanism, the following data sets were obtained and analyzed for any trends between tissue stiffness and bubble dynamics. One topic to note is that, while the following discussion can be translated to experiments involving tissue, gels are the medium in which data was collected. Gels at various concentrations were used to simulate tissues with different viscoelastic properties while still permitting a degree of certainty and consistency that tissue use would not have afforded.

%%%%%%%%%%%%%%%%%%%%%%%%%%%%%%%%%%%%%%%%%%%%%%%%%%%%%%%%%%%%%%%%%%%%%%%%%%%%%%%%%%%%%%%%
\section{Experimental Setup} %%%%%%%%%%%%%%%%%%%%%%%%%%%%%%%%%%%%%%%%%%%%%%%%%%%%%%%%%%%
%%%%%%%%%%%%%%%%%%%%%%%%%%%%%%%%%%%%%%%%%%%%%%%%%%%%%%%%%%%%%%%%%%%%%%%%%%%%%%%%%%%%%%%%
Single bubbles in this experiment were nucleated using two separate methods; a pulsed Nd:YAG laser and a 16-element histotripsy ultrasound array, both focused into the center of a \SI{10}{\cm} diameter sphere. Estimated focal pressure of acoustically-generated single bubbles was 28 MPa. For optical cavitation the laser was operated at a pulse energy of approximately 5 mJ. Minimum thresholds were verified experimentally prior to capturing data, following which the power output by both devices remained relatively constant. This resulted in about half the pulses generating no cavitation, while the other half successfully produced the single bubbles of interest. Blank tests containing no bubble were excluded from the data set. The sphere used as the center of the experiments was 3D printed out of acrylonitrile butadiene styrene (ABS) plastic, with a wide opening in the top to place the gel specimen and four lenses along the equator located orthogonal to one another for the laser, camera, and light source to enter and exit the medium. Materials were suspended in the center of the sphere in water (H$_2$O), cycled via two fluidic ports in the sphere, filtered to \SI{2}{\um} and degassed to \SI{4}{\kPa}. Movement of the gel was done such that a new site of the material was used for each single bubble generated, allowing approximately \SI{3}{\mm} between test sites. Images of the cavitation events were captured using a high speed camera (Phantom) at variable exposure times with a minimum frame rate of \SI{400}{\kHz} and a multi-flash technique utilizing a maximum time between consecutive flashes of \SI{2.5}{\us} to produce a visual of single bubble's rapid growth in a single frame. As shown in Figure \ref{fig:camFlash}, flashes were spaced variably to optimize resolution and distinguish between 2-5 stages of the bubbles initial growth and first collapse, where the bubble is changing at a higher rate compared to the rate of expansion and collapse near maximum radius. By taking this approach we were able to resolve several more radii data points than the maximum frame rate of our camera would allow. Total frames captured per bubble varied as well, capturing a set of frames of the material beginning prior to initial nucleation and continuing until the bubble had finished the bulk of its oscillatory lifetime. A schematic drawing of the experimental setup and flash sequence used is shown in Figure \ref{fig:setup}. Circular approximations of the bubble at each frame estimate the radii values, as evaluated in Figure \ref{fig:Rmax}, for bubbles in water and 0.3\% to 5\% agarose gel. 

\begin{figure}[ht!]
	\begin{center} 
		\includegraphics[width=.48\textwidth]{./figure1.eps} 
	\end{center}
	\caption{Schematic of the experimental setup.}
	\label{fig:setup}
\end{figure}

\begin{figure}[ht!]
	\begin{center} 
		\includegraphics[width=.48\textwidth]{./camFlashDiagram.eps} 
	\end{center}
	\caption{Visualization of multi-flash method and variable timing on images of initial growth and final collapse of a single bubble. Example images of a single bubble generated via intrinsic threshold histotripsy in water.}
	\label{fig:camFlash}
\end{figure}

%%%%%%%%%%%%%%%%%%%%%%%%%%%%%%%%%%%%%%%%%%%%%%%%%%%%%%%%%%%%%%%%%%%%%%%%%%%%%%%%%%%%%%%%%%%%%%%
\section{Results}
%%%%%%%%%%%%%%%%%%%%%%%%%%%%%%%%%%%%%%%%%%%%%%%%%%%%%%%%%%%%%%%%%%%%%%%%%%%%%%%%%%%%%%%%%%%%%%%
The radius of a bubble was found by fitting a circle to the two-dimensional image captured of the bubble and then using that radius as the current dimensional value for each bubble. Example images of bubble lifetimes found in Figure \ref{fig:images} for ultrasound and laser generated bubbles used to acquire radii values in media herby referred to by gel concentration values as it correlates directly to stiffness of the material. All radii and timing values were recorded sequentially in a database from which the following graphical analysis originates. 

\subsection{Bubble Visualization} \label{sec:images}

Clarity proved difficult to obtain in images of higher concentration gels, with the opaque nature of the agarose preventing a significant amount of light from the flash source from reaching the camera. As a result, images obtained were of a lesser resolution than their counterparts in lower gel concentrations. Methodology for producing the higher concentration gels requires refinement, with many of the images containing multiple bubbles that preferentially cavitated at a lower threshold pressure on sites of impurity or contaminants. A bubble's lifetime is also drastically reduced in higher concentration gels, as shown in \ref{fig:vconc}, therefore the maximum frame rate of the camera became a limiting factor in obtaining images capable of being evaluated numerically. For these reasons the data set currently does not contain statistically significant samples of bubble cavitation at gel concentrations over 2.5\%. 

\begin{figure*}[ht!] \centering
	\begin{center} 
		\includegraphics[width=1\textwidth]{./imagestest4.eps} 
	\end{center}
	\caption{Example images highlighting the difference in size of single bubbles generated utilizing both histotripsy and laser in increasingly concentrated gel mediums. Asterisks marks frame(s) outside of the bubbles primary lifetime that were thereby discarded from the database. Frame dimensions = 0.6 x 0.6 mm}
	\label{fig:images}
\end{figure*}

\subsection{Experimentation v. Prediction} \label{sec:exp_v_pred}

Figure \ref{fig:Rmax} summarizes much of the data obtained throughout this experiment, displaying radii values normalized with respect to R$_max$ observed plotted along normalized time with respect to Rayleigh collapse time. Collapse events were largely observed to follow the dynamics as predicted by Raleigh \cite{rayleigh1917} The Rayleigh prediction curve is dashed in each plot, resulting from the following equation: 

\begin{equation}\label{eq:rayPredict}
	\dot{R}=\sqrt{\frac{2P_{\infty}}{3\rho}}\left(\frac{R_m^3}{R^3}-1\right)^{1/2}
\end{equation} 

where $\dot{R}$ is the bubble wall velocity, $P_{\infty}$ is the ambient pressure of the fluid, $\rho$ is the density of the fluid, and $R_m$ and $R$ are the bubble's maximum and instantaneous radii, respectively. Using this standard and the water scenario as a baseline, Figure \ref{fig:Rmax} highlights the deviation from expected bubble dynamics as stiffness of surrounding media increases. 

\begin{figure*}[ht!] \centering
	\begin{center} 
		\includegraphics[width=1\textwidth]{./RTCurvesClean.eps} 
	\end{center}
	\caption{Normalized Radius and time with respect to R$_{max}$ and t$_{Rayleigh}$ respectively of single bubbles in media of varying stiffness.}
	\label{fig:Rmax}
\end{figure*}

\subsection{Bubble Timing and Radii Dependence on Medium} \label{sec:rtcurves}

While it can be seen visually that the maximum achievable radius for a single bubble decreases with increasing gel concentration, as shown in Figure \ref{fig:images}, numerical values for the relationship were found as well, shown in Figure \ref{fig:vconc}. Values for radii at each gel concentration are mean values over the entire data set collected, with error bars corresponding to the combined errors of resolution and standard deviations of each respective gel type. Two power-law fits are displayed as well, done in such a way as to provide a method for interpolating data points at various other gel concentrations while maintaining a high level of accuracy with respect to experimentally obtained statistics.  

\begin{figure}[ht!]
	\begin{center} 
		\includegraphics[width=.48\textwidth]{./rmaxvconc.eps} 
	\end{center}
	\caption{Average values of maximum bubble radius R$_{max}$ compared to gel concentration and cavitation method. Error bars represent standard deviations of R$_{max}$ between experiments of the same gel concentration and method}
	\label{fig:vconc}
\end{figure}

Non-normalized values for time are used in Figure \ref{fig:growCollapse} to display the relationship between expansion/collapse time and gel concentration for both cavitation methods. Time scale starts at \SI{0}{\us} at bubble initiation consistently for all bubbles. Error bars are based on the standard deviation in time values due to the accumulation of tests for each experimental gel set.

\begin{figure}[ht!]
	\begin{center} 
		\includegraphics[width=.48\textwidth]{./growCollapse.eps} 
	\end{center}
	\caption{timing is SOMETHING IDK YET BECAUSE YA}
	\label{fig:growCollapse}
\end{figure}

%%%%%%%%%%%%%%%%%%%%%%%%%%%%%%%%%%%%%%%%%%%%%%%%%%%%%%%%%%%%%%%%%%%%%%%%%%%%%%%%%%%%%%%%%%%%%%%
\section{Discussion} %%%%%%%%%%%%%%%%%%%%%%%%%%%%%%%%%%%%%%%%%%%%%%%%%%%%%%%%%%%%%%%%%%%%%%%%%%
%%%%%%%%%%%%%%%%%%%%%%%%%%%%%%%%%%%%%%%%%%%%%%%%%%%%%%%%%%%%%%%%%%%%%%%%%%%%%%%%%%%%%%%%%%%%%%%
In this study we have observed a number of interesting yet predictable features associated with bubble dynamics in a variety of environments, including the inverse relationship between bubble volume, time and stiffness. 

Images began being captured at the start of each nucleation at a consistent rate of \SI{2.5}{\us}, creating a timing error in the data set. Due to this limitation on precision in the image set obtained, the possibility arised that the true maximum radius was not captured. By scaling and shifting individual radius versus time curves within said timing error and coellescing all experiments involving a specific medium to a single plot, we observed that these curves display the relationship between gel concentration and deviation from Ralyeigh collapse time consistently in a statistically significant manner (Figure \ref{fig:Rmax}). Correlation between the normalized variables of interest became weak as gel concentration increased, lessening the statistical significance of observations but still permitting conclusions to be drawn relating theoretical and experimental curves. For gel concentrations above 0.3\%, experimental values for timing began to deviate from those predicted by Rayleigh. In 1.0\% we discern a growth and collapse time value that is approximately 80-90\% that of t$_{Rayl,Collapse}$, decreasing to 70\% t$_{Rayl,Collapse}$ in 2.5 and 5\% gel concentrations. Difficulties arise when an effort is made to draw substantial conclusions on the difference between growth and collapse times due to the alterations (scaling, shifting, normalization) made to Figure \ref{fig:Rmax}. 

Figure \ref{fig:growCollapse} utilizes non-normalized values for time in an effort to observe any symmetry or dissymmetry in timing. (ADD STUFF TO THIS ONCE J FIGURES OUT STATISTICAL SIGNIFICANCE)

Throughout the study it was clear an increase in gel concentration resulted in a decrease in the maximum bubble radii of single bubbles nucleated regardless of method, an observation validated by the data set collected in Figure \ref{fig:vconc}. Both laser and histotripsy induced bubbles decreased with respect to a power law trend, providing a mathematical relationship between bubble size and gel concentration that can be used in further data collection and to predict a bubble's approximate size due to the medium in which it is formed. One possible explanation for the differences in values of maximum radius for histotripsy and laser nucleated bubbles in both water and 0.3\% gel is the variations in the cavitation mechanism. Laser nucleated cavitation is a result of internal positive pressure spikes at the focus of the laser beam, while intrinsic threshold histotripsy cavitates due to spikes in negative external pressure at the focus \cite{sukovich2011pressure}. In environments with higher elastic moduli, bubbles tended to proceed with similar trends regardless of nucleation method (Figure \ref{fig:vconc}). Laser nucleation continuously resulted in marginally larger bubble radii, an observation we attribute to energy deposition differences between laser and histotripsy rather than initial nucleation method. 

Model verification was a key motivation in these experiments. Currently there are many models used to evaluate collapse of single bubbles in different materials; however, models of growth can be complex and in general are lacking when compared to their collapse counterparts. Here we present a quantitative analysis of data gathered experimentally with the hopes of benefitting future models pertaining to both the growth and collapse of cavitation-driven bubbles. MODEL DISCUSSION IF WE GET THAT
%%%%%%%%%%%%%%%%%%%%%%%%%%%%%%%%%%%%%%%%%%%%%%%%%%%%%%%%%%%%%%%%%%%%%%%%%%%%%%%%%%%%%%%%%%%%%%%
\section{Conclusion} %%%%%%%%%%%%%%%%%%%%%%%%%%%%%%%%%%%%%%%%%%%%%%%%%%%%%%%%%%%%%%%%%%%%%%%%%%
%%%%%%%%%%%%%%%%%%%%%%%%%%%%%%%%%%%%%%%%%%%%%%%%%%%%%%%%%%%%%%%%%%%%%%%%%%%%%%%%%%%%%%%%%%%%%%%
Overall these experiments serve as a starting point for further testing, with the goal of using bubble dynamics to quantify the rheological properties of different materials and how viscosity and bubble growth changes with respect to elastic modulus. To build on this foundation for modelling and gather an increasing amount of practical data, further testing may take several different directions. Perhaps the simplest, and most impactful, would be to repeat the experiment in a variety of other transparent gel types, including polyacrylamide gel at varying concentrations. Agar gel has a very specific uniform molecular structure and, while useful for this data set, does not encompass all possible viscoelastic behaviors present in human tissue. One restriction to address in future experiments is it becomes progressively more difficult to reliably generate single, spherical bubbles using histotripsy as gel concentration is increased and nearly impossible (less than 1\%) at gel concentrations at or above 5\%. Replication of this experimental setup and execution using different materials will provide a larger pool of data pertaining to bubble growth and collapse which the modelling community can use to further approximate bubble dynamics in materials of interest. 

\bibliography{Paper.bib}{}
\bibliographystyle{ieeetr}

\end{document}
